\chapter{Conclusão}
\label{chap:conclusao}

    Neste trabalho, apresentamos um quadro formal para auxiliar desenvolvedores no processo de criação e teste de sistemas complexos, mais especificamente em sistemas de inteligência ambiental. Sistemas \acrshort{ami} são aplicações que utilizam dispositivos distribuídos por um ambiente que, através de programas de \textit{software} inteligentes, são capazes de fornecer uma série de serviços para os usuários. 
    
    São diversas as possibilidades de aplicações em sistemas \acrshort{ami} e com a popularização tanto dos dispositivos móveis (celulares, \textit{tablets}, \textit{smartwatches}) e dos sensores inteligentes, esses sistemas se tornam cadas vez mais comuns. Grandes empresas já possuem sistemas que utilizam \textit{smartphones} e outros sensores instalados nas casas dos usuários para prover uma série de facilidades aos moradores, variando desde um sistema de entretenimento comandado por voz, até sistemas de segurança capazes de reconhecer os moradores da casa. 
    
    A construção de ambientes inteligentes, no entanto, ainda é processo bastante complexo que pode envolver um conjunto muito grande de artefatos. Esses artefatos, dispositivos de \textit{hardware} e \textit{software}, interagem entre si e entre um grupo de pessoas no ambiente. Tal complexidade torna-se ainda maior quando consideramos que esse tipo de sistema só pode ser testado após a instalação do \textit{hardware} necessário no ambiente. Ou seja, grande parte dos erros relacionados à estrutura do projeto, e até mesmo do código, só será descoberto após a sua construção, elevando ainda mais os custos de produção. 
    
    A abordagem proposta neste trabalho, considera que o sistema \acrshort{ami} a ser construído deverá ser controlado por um Sistema Multiagente, onde cada agente é capaz de controlar um ou mais dos dispositivos espalhados pelo ambiente. A utilização de \acrshort{sma}s permite uma melhor utilização das características distributivas da inteligência ambiental, além de possibilitar uma construção individual de cada entidade. 
    
    A utilização de \acrshort{sma}s possibilita  também a representação de cada entidade do sistema como um conjunto de agentes. Neste trabalho, cada agente que representava uma entidade do sistema técnico da aplicação de inteligência ambiental é representado em uma simulação por um conjunto contendo três agentes que atuam nos papéis de sensor, atuador e controlador. Essa visualização das entidades do sistema como um conjunto de agentes, juntamente com o modelo \acrshort{fec} proposto neste trabalho, permite ao desenvolvedor criar simulações de sistemas \acrshort{ami} sendo capaz de controlar diversos aspectos do ambiente e do próprio \textit{hardware} que compõe o dispositivo.
    
    O modelo \acrshort{fec} proposto neste trabalho divide o sistema em três partes. Na primeira, denominada modelo de Função, são definidas as entidades (agentes) que fazem parte da aplicação, suas percepções e ações. Nela também definimos o objetivo original do sistema e seus objetivos próximos, determinando um conjunto de funções capazes de medir a satisfação do sistema em relação ao alcance desses objetivos. No modelo de Estrutura são determinados os relacionamentos entre os agentes criados, definindo os tipos de mensagens que podem ser enviadas e estipulando uma frequência de comunicação. Por último, no modelo de Comportamento são elaborados os esqueletos dos programas dos agentes da aplicação. Neste ponto são definidas as estratégias que serão utilizadas a fim de alcançar os objetivos definidos no modelo F. 
    
    Com os modelos \acrshort{fec} definidos, podemos construir uma simulação da aplicação de inteligência ambiental. Essa simulação deverá ser capaz de mostrar ao desenvolvedor do sistema as falhas nas estratégias utilizadas, permitindo a ele corrigir os modelos e testar novas estratégias ainda em tempo de projeto.  
    
    Para validar o modelo \acrshort{fec} proposto, elaboramos uma aplicação \acrshort{ami} simples que utiliza um grupo de agentes aspiradores de pó para limpar um ambiente onde circulam um grande número de pessoas. A simulação gerada a partir desse modelo foi construída utilizando a plataforma NetLogo \cite{wilensky1999netlogo}, cujos códigos podem ser acessados no seguinte endereço \footnote{https://bitbucket.org/rob\_oliveira/netlogocleanersim}. A aplicação é estudada ao longo de vários experimentos, cada um deles é gerado a partir de modificações nos modelos \acrshort{fec}. A cada experimento são extraídos dados relacionados a performance do sistema; essas informações são construídas com base no modelo F proposto. 
    
    O desempenho do sistema, observado no Capítulo \ref{cap:resultados}, mostra a utilidade da aplicação em cada um dos cenários. Os três primeiros experimentos têm como objetivo mostrar que o agente criado é minimamente capaz de atingir os objetivos do sistema independente dos obstáculos inseridos no ambiente. Os três últimos experimentos ilustram estratégias diferentes que a aplicação pode utilizar para atingir os objetivos. No primeiro são inseridos diversos agentes aspiradores no ambiente, eles interagem com os agentes usuários e o próprio ambiente, buscando alcançar os objetivos definidos. No segundo, inserimos um agente coordenador, capaz de delegar tarefas aos demais agentes. E, no último, os agentes são separados em grupos, diminuindo sua areá de atuação. 
    
    Apesar das diferenças entre as estratégias utilizadas, as performances observadas no fim dos experimentos foram semelhantes, embora inferiores às esperadas. Entre os três últimos experimentos, o experimento \ref{sec:multi-asp}, apesar de mais simples, foi o que obteve os melhores resultados, demonstrando que as estratégias utilizadas nos experimentos \ref{sec:exp-coord} e \ref{sec:grupo}, ou o modelo de função proposto, devem ser repensadas. 
    
    Mesmo com um desempenho abaixo do esperado, podemos considerar que os experimentos foram um sucesso, tendo em vista que o objetivo do trabalho é validar o modelo \acrshort{fec} proposto. Com a construção das simulações através dos modelos, o desenvolvedor da aplicação pode testar diversas estratégias para a resolução de um determinado problema de forma rápida, sem a necessidade dos dispositivos físicos e com um custo baixo.
    
    \section{Trabalhos Futuros}
    Como este trabalho apresenta um novo quadro formal para a construção de sistemas e simulações, ainda devem existir diversos trabalhos evoluindo e diversificando os conceitos apresentados. Como primeiro passo, é necessário evoluir os modelos elaborados aqui, simplificando-os e tornando-os mais flexíveis para diferentes tipos de sistema. Para isso, também devemos testar o modelo \acrshort{fec} em uma aplicação real para validar, de fato, a sua eficiência. 
    
    Um outro trabalho possível é a criação de uma ferramenta capaz de modelar sistemas seguindo os conceitos apresentados neste trabalho. A partir desses modelos, poderíamos gerar códigos para a criação de simulações em linguagens como NetLogo, aumentando ainda mais a velocidade de desenvolvimento. 
    