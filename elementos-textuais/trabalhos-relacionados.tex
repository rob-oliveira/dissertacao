\chapter{Trabalhos Relacionados}
\label{cap:trabalhos-relacionados}


\citeonline{parunak1998agent} comparadam duas formas de modelagem para a simulação de um sistema de cadeia de suprimentos. A primeira utiliza uma série de fórmulas matemáticas para simular o comportamento do sistema com o passar do tempo. A segunda utiliza  modelagem baseada em agentes para construir virtualmente o sistema e observar seu funcionamento. Como conclusão, o autor define que para sistemas centralizados é mais apropriado o uso de modelagens com base matemática. Dessa forma, a modelagem baseada em agentes deve ser usada em sistemas com alto grau de distribuição.

\citeonline{norling2000enhancing} utiliza modelagens baseadas em agentes para construir simulações da sociedade humana capazes de exibir comportamentos mais realistas. Para alcançar esse resultado, os agentes que compõem a simulação utilizam um mecanismo de tomada de decisão chamado \textit{NDM (Naturalistic Decision Making)}, que leva em consideração diversos fatores além da racionalidade para ao tomar uma decisão. 

\citeonline{davidsson2000multi} realiza um estudo sobre o uso de modelagens centralizadas e descentralizadas no processo de criação de simulações. Para ilustrar o estudo são demonstrados  os processos para uma simulação de sistema de desenvolvimento de software tanto de forma centralizada quanto descentralizado, comparando cada uma das metodologias. Gerando um conjunto de diretrizes que auxiliam o uso de simulações baseadas em agentes.
 
\citeonline{davidsson2005distributed} utilizam \acrshort{mabs} para modelar e simular um sistema que monitora e controla um escritório em um prédio comercial. O grande diferencial da abordagem proposta nesse trabalho, é que todos os componentes do sistema (as pessoas, o hardware, e o próprio software do sistema),  são representados na modelagem por agentes inteligentes. Utilizando essa abordagem ele define diversos comportamentos para cada um dos agentes, apresentando o desempenho do sistema com essa variação.

\citeonline{mustafa2013simulation} descreve o processo de desenvolvimento de um simulador para aplicações em \acrfull{aal}. Assim como aplicações \acrshort{ami}, sistemas de vivência assistida geram grandes fluxos de informação e dependem de sensores espalhados em um determinado ambiente. A aplicação descrita no trabalho utiliza um conjunto de regras sobre uma base de dados para gerar um sistema de inferência. Essas regras são determinadas por especialistas mas o sistema é capaz de armazenar os dados coletados pelos sensores e utilizá-los para adaptar a aplicação às necessidades de cada usuário. 

\citeonline{macal2014introductory} apresentam um conjunto de boas práticas relacionadas ao desenvolvimento de sistemas utilizando modelagem baseada em agentes. Ele descreve os contextos onde essa técnica pode ser utilizada, os tipos de agentes modelados, e os tipos de relacionamentos elaborados a partir deles.

\citeonline{francoempirical} trata de um problema de organização e design de times de agentes em um ambiente de competição onde cada um dos times deve disputar os recursos disponíveis em um ambiente. Apesar da abordagem do trabalho ser diferente, o problema é bastante similar ao enfrentado ao elaborar projetos de sistemas \acrshort{ami} onde, dependendo da aplicação, podemos ter diversas entidades com objetivos conflitantes (software vs hardware, ou software vs usuário).

\citeonline{grossi2007structural, grossi2005foundations} estabelecem um conjunto de técnicas que auxiliam na avaliação de organizações multiagentes.Em seus trabalhos são definidos diversos conceitos, baseados na teoria dos grafos, para extrair características como robustez, flexibilidade, e eficiência de uma determinada organização. Essas características são utilizadas para comparar diferentes organizações quanto ao seu desempenho para a resolução de um determinado problema.

\citeonline{barry2009agent} defendem que a utilização de simulações dirigidas por agentes (\acrshort{ads}) é essencial para a construção de sistemas complexos. A aplicação de \acrshort{ads} em conjunto com técnicas de engenharia de sistemas pode prover uma maior compreensão do sistema como um todo, facilitando seu design, e melhorando sua performance. 

Dentre as vantagens da utilização de \acrshort{ads} no processo de desenvolvimento \citeonline{barry2009agent} destaca as seguintes: (1) pode ser usada como uma ferramenta de geração de requerimentos; (2) permite prototipar e analisar funções do sistema; (3) pode ser usada como plataforma de visualização; (4) um componente pode ser usado para representar vários outros, modificando certas características para dar mais realismo a simulação; (5) o experimento pode ser realizado nos mais diversos contextos, ilustrando desde cenários ideais, até ambientes catastróficos para o sistema. 

\citeonline{hubner2002model, hubner2007developing} descrevem a linguagem de modelagem organizacional Moise, que representa formalmente uma organização de agentes. Essa organização pode ser descrita através da especificação organizacional, \acrfull{os}, que pode ser dividida em três dimensões, \acrfull{ss}, \acrfull{fs}, \acrfull{ds}. A especificação estrutural (\acrshort{ss}) determina os papéis e seus relacionamentos, além dos grupos que compõem a organização. A especificação funcional (\acrshort{fs}) define os objetivos globais e o modo como eles serão alcançados. E por fim, a especificação deôntica (\acrshort{ds}) relaciona essas duas dimensões, identificando qual conjunto de objetivos e funções da camada funcional será atribuído a cada papel da camada estrutural. 

%%%intituições Eletronicas

\citeonline{rodriguez1999towards} definem  em seu trabalho sobre \acrfull{ei}um protocolo capaz de representar sistemas complexos através de maquinas de estado. Em uma Instituição Eletrônica cada estado simboliza uma cena e cada cena possui um conjunto de estados que representam as entidades que compõem o sistema. A utilização de máquinas de estado permite a automatização do processo de verificação das propriedades de um determinado estado. Cada uma ilustra um aspecto do sistema que será desenvolvido, detalhando as entidades envolvidas e as relações entre elas. 

%\section{Metodologia AVA}
%\label{sec:ava}

\citeonline{garcia2010human} definiram a metodologia AVA que é um conjunto de procedimentos que guiam o desenvolvedor na definição, criação, teste, e validação do sistema \acrshort{ami}.  O foco deste trabalho é auxiliar na criação de ambientes inteligentes através de um paradigma de simulação onde cada componente é estudado separadamente. A simulação é criada a partir de uma modelagem de sistemas multiagentes, onde cada agente é especificado de acordo com um componente e suas interações.

A partir da definição dos modelos do ambiente e dos usuários, são instanciados os modelos do sistema \acrshort{ami} a ser simulado. Com a implementação desses modelos, são executadas diversas simulações e em seguida seus resultados são analisados afim de detectar \textit{bugs} ou melhorias possíveis no sistema. Cada mudança no sistema reinicia o ciclo da metodologia, refatorando a modelagem da aplicação, seus modelos e suas implementações. Esse ciclo deve continuar até o ponto onde o desenvolvedor considera que o sistema é estável. 

Um outro conceito Bastante interessante inserido neste trabalho é a noção de inserção de realidade. Se o desenvolvedor da aplicação não encontrar erros na simulação ele pode inserir elementos do mundo real na simulação de forma gradual. Controlando cada etapa do sistema. A cada novo elemento "injetado" o ciclo de é repetido tornando o sistema cada vez mais robusto. Esse processo pode continuar até o ponto onde todo o sistema tenha sido implantado no mundo real. 

\section{Considerações}

O trabalho de \citeonline{parunak1998agent} é utilizado como base para conceitos empregados não só neste trabalho mas também em vários dos outros trabalhos citados neste capítulo. A modelagem de sistemas multiagentes (\acrshort{mabs}) é um conceito fundamental dos trabalhos de \citeonline{davidsson2005distributed} que serviram como a inspiração inicial deste projeto. Nesses trabalhos, Davidsson utiliza \acrshort{mabs} para modelar e simular não só sistemas mas também seus usuários e os equipamentos que o compõem. O trabalho de \citeonline{norling2000enhancing} também utiliza esse tipo de modelagem para simular pessoas. Porém, diferente de Davidsson, seu objetivo não é o aperfeiçoamento do sistema mas sim gerar agentes com um comportamento mais humano.

Dentre as linguagens de modelagem para \acrshort{sma} a linguagem \textit{Moise} \cite{hubner2002model, hubner2007developing} descreve formalmente organizações de agentes através de três modelos que se complementam. No trabalho de \citeonline{rodriguez1999towards} os \acrshort{sma} são descritos através das instituições eletrônicas, capazes de definir os protocolos de comunicação entre cada entidade. A linguagem Moise, apesar de fornecer uma excelente representação da organização de agentes, não ilustra com naturalidade as interações entre as entidades. Por outro lado, as instituições eletrônicas explicitam os protocolos de comunicação entre os agentes de uma organização, mas podem ser bastante complexas para certos tipos de sistema. 

A metodologia AVA \cite{garcia2010human} define um conjunto de procedimentos que guiam o desenvolvedor na definição, criação, teste e validação de sistemas \acrshort{ami}. Para isso, ela utiliza uma série de simulações baseadas em sistemas multiagente, que aos poucos evoluem o sistema até o ponto em que o desenvolvedor considere suficiente para inserir elementos reais. A inserção de componentes reais do sistema na simulação é chamada de inserção de realidade, e permite que, aos poucos,  os componentes da simulação sejam substituídos por elementos do mundo real, até o ponto em que o sistema esteja totalmente implantado no mundo em seu ambiente físico.

Apesar de auxiliar em diversas etapas da construção e simulação de sistemas \acrshort{ami}, a metodologia AVA não define uma abordagem para determinar os comportamentos dos agentes que compõem o sistema ou a simulação. A abordagem nesta dissertação utiliza conceitos propostos em \acrshort{mabs} para definir um processo de simulação de sistemas, focando em sistemas \acrshort{ami}, utilizando como base alguns dos conceitos propostos nos trabalhos de \citeonline{hubner2007developing} e \citeonline{rodriguez1999towards}. Definindo um modelo comportamental, estrutural e funcional para os agentes do sistema. 


