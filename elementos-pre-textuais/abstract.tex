Construction of large systems is a complex activity that requires a lot of planning, even more in the project initial stages. Errors in this stage disseminate throughout the application lifecycle, increasing its cost. As the system grows, it becomes more complex and more error-prone. This is due to a large number of components the system has and the high degree of interaction between them.
Ambient Intelligence (AmI) applications use several devices, spread throughout the environment to interact with users. In small environments with a low movement of people, a developer could build an AmI project with little concern related to project errors, but as the environment grows, more users arrive and more devices are used, becoming harder to establish the exact behaviour of the system.
This work objective is to design a computational approach capable of contributing to the creation of complex sociotechnical systems in the design and performance evaluation stages. We intend to define a computational approach, capable of helping the system designer to better understand the system he intends to construct, defining its functions, structure, and behaviour.
In order to help the developer visualize the results of his vision, defined according to our approach, we propose the use of multi-agent based simulations. The purpose of this simulation is to show the developer how the system will behave before the construction process is started, allowing the detection of possible project failures.
%As preliminary results of this work we present the formalizations of function, structure and behavior required for the construction of the system according to our approach.

% Separe as Keywords por ponto
\keywords{Ambient Intelligence. AmI. Multi-Agent Based Simulatuions.}