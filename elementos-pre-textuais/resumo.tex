
A construção de grandes sistemas é uma atividade complexa que exige um alto nível de planejamento, ainda mais nas etapas iniciais do projeto. Erros nessa etapa se propagam durante todo o ciclo de vida do sistema, e podem aumentar seu custo ou até mesmo inviabilizá-lo. Quanto mais complexo o sistema, maiores são as chances de o projeto apresentar erros. Isso ocorre devido ao grande número de componentes que o sistema possui e ao alto grau de interação entre eles. 
Em aplicações de Ambientes Inteligentes (AmI), por exemplo, temos, em um determinado espaço físico, diversos dispositivos inteligentes, capazes de interagir com o ambiente e com usuários através de sensores e atuadores. Em ambientes pequenos e com pouca circulação de pessoas, o desenvolvedor poderia criar um projeto sem muitas dificuldades, mas, à medida que o ambiente ganha volume, o número de pessoas circulando cresce e a quantidade de dispositivos aumenta, ficando mais difícil determinar o comportamento exato do sistema. 
O objetivo deste trabalho é idealizar uma abordagem computacional capaz de contribuir com a criação de sistemas sociotécnicos complexos nas etapas de projeto e de avaliação de desempenho. Ou seja, pretendemos definir uma abordagem capaz de auxiliar desenvolvedores de sistema \textit{AmI} a entender melhor o que eles pretendem construir, definindo suas funções, sua estrutura, e seu comportamento. 
A abordagem desenvolvida utilizará programas de agentes inteligentes, onde cada agente deverá pertencer a uma organização que representa uma entidade do sistema. Os agentes, além de serem capazes de representar as funcionalidades das entidades, devem fornecer ao desenvolvedor informações quanto ao seu desempenho dentro da organização. Além disso, para auxiliar o desenvolvedor a visualizar os resultados da sua visão do sistema, propomos também a utilização de simulações baseadas nos agentes definidos. O objetivo dessa simulação é mostrar ao desenvolvedor como o sistema irá se comportar antes que o processo de implementação seja iniciado, permitindo a detecção de possíveis falhas do projeto.
%Como resultados preliminares desse trabalho apresentamos as formalizações do modelo FEC proposto, necessárias para a construção do sistema de acordo com nossa abordagem. 

\palavraschave{Ambientes Inteligentes. AmI. Simulações Baseadas em Multiagentes.}